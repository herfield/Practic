 \documentclass[12pt,a4paper]{scrartcl} 
\usepackage[utf8]{inputenc}
\usepackage[english,russian]{babel}
\usepackage{indentfirst}
\usepackage{misccorr}
\usepackage{graphicx}
\usepackage{amsmath}
\begin{document}
{\LARGE \textit{Вариант 2}}\bigskip

{\LARGE \textit{Тема: Генератор случайных чисел Парка-Миллера с перетасовкой и без.}}
			
\section{Ход работы}
\label{sec:exp}

\subsection{Код приложения}
\label{sec:exp:code}
\begin{verbatim}
//Генератор случайных чисел Парка-Миллера с перетасовкой

/*#include <stdio.h>
#include <iostream>
#define a 16807
#define m 2147483647
#define am (1.0/m)
#define q 12773
#define r 2836
#define n 32
#define w 8
#define d (1+(m-1)/n)
#define e 1.2e-7
#define x (1.0-e)
#define MASK 123456789

static long t;

void Seed(long t1)
{
	t = t1;
}

float unirand0(void)
{
	long k;
	float ans;

	t ^= MASK;
	k = t / q;

	if ((t = a * (t - k * q) - r * k) < 0)
		t += m;

	ans = am * t;

	t ^= MASK;

	return(ans);
}

float unirand1(void)
{
	int z;
	long k;
	static long y = 0, v[n];
	float temp;


	if (t <= 0 || !y)
	{
		if (t < 0)
			t = -t;
		else if (t == 0)
			t = 1;

		for (z = n + w - 1; z >= 0; z--)
		{
			k = t / q;

			if ((t = a * (t - k * q) - r * k) < 0)
				t += m;

			if (z < n)
				v[z] = t;
		}


		y = v[0];
	}

	k = t / q;
	if ((t = a * (t - k * q) - r * k) < 0)
		t += m;

	y = v[z = y / d];
	v[z] = t;


	if ((temp = am * y) > x)
		return(x);
	else
		return(temp);
}
using namespace std;
int main()
{
	setlocale(LC_ALL, "ru");
	cout << "Генератор случайных чисел Парка-Миллера c перетасовкой" << endl;
	int i;
	Seed(6723);
	for (i = 0; i < 100; i++)
		printf("%f\n", unirand1());
}*/

//Генератор случайных чисел Парка-Миллера без перетасовки

#include <stdio.h>
#include <iostream>
#define a 16807
#define m 2147483647
#define am (1.0/m)
#define q 12773
#define r 2836
#define MASK 123456789

static long t;

void Seed(long t1)
{
	t = t1;
}

float unirand(void)
{
	long k;
	float ans;

	t ^= MASK;
	k = t / q;

	if ((t = a * (t - k * q) - r * k) < 0)
		t += m;

	ans = am * t;

	t ^= MASK;

	return(ans);
}
using namespace std;
int main()
{
	setlocale(LC_ALL, "ru");
	cout << "Генератор случайных чисел Парка-Миллера без перетасовки" << endl;
	int i;
	Seed(6723);
	for (i = 0; i < 100; i++)
		printf("%f\n", unirand());
}
\end{verbatim}


\section{Код в работающем состоянии}
\label{sec:picexample}
\begin{figure}[h]
	\centering
	\includegraphics[width=0.4\textwidth]{11.jpeg}
	\caption{Генератор случайных чисел с перетасовкой}\label{fig:par}
\end{figure}
\label{sec:picexample}
\begin{figure}[h]
	\centering
	\includegraphics[width=0.4\textwidth]{22.jpeg}
	\caption{Генератор случайных чисел без перетасовки}\label{fig:par}
\end{figure}

\section{Библиографические ссылки}

Для изучения «внутренностей» \TeX{} необходимо 
изучить~\cite{andreyolegovich}, а для изучения Get лучше
почитать~\cite{proglib.io}.Чтобы понять как работает генератор случайных чисел Парка-Миллера, нужно обратится к~\cite{q}. 

\begin{thebibliography}{9}
\bibitem{andreyolegovich}Изучение \LaTeX{}. https://www.andreyolegovich.ru/PC/LaTeX.phpbase
\bibitem{proglib.io}Изучение Get. https://proglib.io/p/git-for-half-an-hour/    
\bibitem{q}Генератор СЧ Парка-Миллера. https://algolist.manual.ru/maths/generator/parkmil.php
\end{thebibliography}

\end{document}